\documentclass[11pt,a4paper]{article}
\usepackage{bbm,amsthm,amsfonts,amssymb,amsmath,latexsym,epic,eepic}
\usepackage{marvosym,graphicx,fancyhdr,bbm}
\usepackage[dvips]{color}
\usepackage[rflt]{floatflt}
\usepackage{colortbl}
\definecolor{Grey}{rgb}{0.5,0.5,0.5}
\definecolor{Red}{rgb}{1.0,0.0,0.0}

\usepackage{typearea}
\areaset{156mm}{235mm}
%\setlength{\parskip}{5pt plus 2pt minus 1pt}
\setlength{\parindent}{0pt}

% use \M for matrices and \V for vectors in math mode
\newcommand{\M}[1]{\mathbf{#1}}
\newcommand{\V}[1]{\mathbf{#1}}
\newcommand{\norm}[1]{\left | \left | #1 \right | \right |}
\newcommand{\RR}{\mathbbm{R}}        % set of real numbers


\renewcommand\floatpagefraction{0.8}
\renewcommand\topfraction{1}
\renewcommand\bottomfraction{0.9}
\renewcommand\textfraction{0.0}
%\def\dbltopfraction{1.0}
%\def\bottomfraction{1.0}
%\def\dblfloatpagefraction{0.8}


\makeatletter
\renewenvironment{thebibliography}[1]
     {\section*{\refname}%
      \@mkboth{\MakeUppercase\refname}{\MakeUppercase\refname}%
	 \parsep0mm
	 \itemsep0mm
	 %\labelsep0mm
	 %\itemindent0mm
      \list{\@biblabel{\@arabic\c@enumiv}}%
           {\settowidth\labelwidth{\@biblabel{#1}}%
            \leftmargin\labelwidth
            \advance\leftmargin\labelsep
            \@openbib@code
            \usecounter{enumiv}%
            \let\p@enumiv\@empty
            \renewcommand\theenumiv{\@arabic\c@enumiv}}%
      \sloppy
      \clubpenalty4000
      \@clubpenalty \clubpenalty
      \widowpenalty4000%
      \sfcode`\.\@m}
     {\def\@noitemerr
       {\@latex@warning{Empty `thebibliography' environment}}%
      \endlist}
\renewcommand\newblock{\hskip .11em\@plus.33em\@minus.07em}
\let\@openbib@code\@empty
\makeatother



\begin{document}



\title{\Large\bf Vergleich der Pfadverfolgung mit Odometrie und AMCL \footnotetext{Diese Arbeit ist Bestandteil des Praktikums zur Mess- und Regelungstechnik}}

\author{Kai Hofmann und Barbara Fischbach\\
  Robotik und Telematik \\
  Universit\"at W\"urzburg\\
  Am Hubland, D-97074 W\"urzburg\\
{\small \texttt{barbara.fischbach@uni-wuerzburg.de}}\\
{\small \texttt{kai.hofmann@uni-wuerzburg.de}}}

\date{}




\maketitle

\newpage

\tableofcontents{}

\newpage

\twocolumn

\section*{Abstract}
\addcontentsline{toc}{section}{Abstract}
{Die Automatisierung von Fahrzeugen spielt heutzutage immer eine gr\"o\ss{}ere Rolle. Nicht nur im Weltall, wo wir unbedingt darauf ange-wiesen sind, dass Systeme autonom funktionieren, sondern auch auf der Erde, um Systeme sicherer und bequemer f\"ur den Benutzer zu machen. Um das überaus komplexe Thema verständlicher zu machen und Algorithmen vorstellbar zu erklären, wird im folgenden ein Fraunhofer-Roboter benutzt, der anhand Odometrie,... sich selbständig in einem bekannten Raum zurechtfinden kann.}


\section{Einleitung}

\subsection{title}
\subsubsection{title}
\subsubsection{title}

\section{Odometrie}
{
	}
\section{AMCL}

\section{Odometrie und AMCL im Vergleich}
insert Diagramm mit Position durch Odometrie und AMCL 

\section{Pfadverfolgung mit Giovanni Indiveri}
\section{Zusammenfassung und Ausblick}


{%\small                   % use small if you need it
\bibliographystyle{plain}
\bibliography{paper}       % use a bib-file paper.bib to collect your references
}

\end{document}

